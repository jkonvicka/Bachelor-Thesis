\chapter{Úvod}
\label{sec:Introduction}
Oblast vysoce výkonného počítání (HPC) se neustále rozrůstá, navyšují se kapacity, prostředky a s tím i narůstají možnosti tyto služby využít na stále sofistikovanější problémy. I přes tento vývoj je výpočetní kapacity trvalý nedostatek, uživatelé jsou tak omezováni např. plánovačem nebo počtem jádrohodin, které mohou v rámci svého projektu využít.

Vývojové týmy v superpočítačových centrech se snaží maximálně zjednodušit přístup ke koncovým výpočetním uzlům, aniž by došlo ke snížení výkonnosti. V IT4Innovations se jeden z výzkumných týmů zabývá vývojem platformy, která uživateli umožňuje k superpočítači přistupovat jako ke službě (HPC-as-a-Service). Tato platforma nese označení HEAppE Middleware (High-End Application Execution Middleware) a mezi její stěžení funkce patří vytváření, spouštění a získávání stavu úlohy. Účelem tohoto projektu je vytvoření aplikačního rozhraní se všemi funkcionalitami, které poskytuje plánovač na superpočítačovém clusteru. 

Cílem této práce je již zmíněný middleware (HEAppE Middleware) rozšířit o možnost spouštět testovací úlohy lokálně u uživatele bez nutnosti napojení na fyzický výpočetní cluster a umožnit uživateli vytvořit jeho vlastní šablonu pro spouštění úlohy. Praktickým využitím bude pak simulace HPC úlohy přímo na počítači koncového uživatele. Až bude s návrhem své šablony spokojen a vše si lokálně otestuje, tak si bude moct jednoduše danou úlohu spustit na skutečném výpočetním clusteru.

Součástí práce je i vytvoření testovacích plánů, do kterých bude zahrnuto i testování výše uvedených rozšíření HEAppE Middleware.
\endinput