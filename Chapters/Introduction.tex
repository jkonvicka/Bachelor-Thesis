\chapter{Úvod}
\label{sec:Introduction}
Oblast vysoce výkonného počítání (HPC) se neustále rozrůstá. Navyšují se kapacity, prostředky, \\a s tím narůstají i možnosti tyto služby využít na stále sofistikovanější problémy. I přes tento vývoj je výpočetní kapacity trvalý nedostatek, uživatelé jsou tak omezováni např. plánovačem nebo počtem jádrohodin, které mohou v rámci svého projektu využít.

Vývojové týmy v superpočítačových centrech se snaží maximálně zjednodušit přístup ke koncovým výpočetním uzlům, aniž by došlo ke snížení výkonnosti. V IT4Innovations se jeden z výzkumných týmů zabývá vývojem platformy, která uživateli umožňuje k superpočítači přistupovat jako ke službě (HPC-as-a-Service). Tato platforma nese označení HEAppE Middleware (High-End Application Execution Middleware) a mezi její stěžejní funkce patří vytváření, spouštění a získávání stavu úlohy ze superpočítačového clusteru. Smyslem tohoto projektu je vytvoření rozhraní za účelem snadného přístupu k HPC a „odstínění“ standardních uživatelů od specifických funkcionalit. Rozhraní HEAppE je pak dále integrováno do dalších aplikací.

Cílem této práce je již zmíněný middleware (HEAppE Middleware) rozšířit o možnost spouštění testovací úlohy lokálně na počítači uživatele bez nutnosti napojení na fyzický výpočetní cluster \\a umožnit uživateli spravovat jeho vlastní šablony pro spouštění úloh. Praktickým využitím bude pak simulace HPC úlohy přímo na počítači koncového uživatele. Až bude s návrhem své šablony spokojen a vše si lokálně otestuje, bude si moct jednoduše danou úlohu spustit na skutečném výpočetním clusteru.

Součástí práce je i vytvoření testovacích plánů, do kterých bude zahrnuto i testování výše uvedených rozšíření HEAppE Middleware.
\endinput