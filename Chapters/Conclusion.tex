\chapter{Závěr}
Při návrhu a implementaci podpory simulovaného plánovače spouštěného na lokálním stroji uživatele byla vytvořena konfigurace virtuálního stroje, jehož cílem je simulace chování superpočítačového clusteru s plánovačem úloh. Důraz byl kladen také na co možná nejvěrohodnější simulaci chování a komunikaci plánovače na lokálním virtuálním clusteru. Virtuální stroj byl navržen tak, aby jej uživatel mohl jednoduše upravit či zaměnit za své řešení. Musí přitom však dodržovat předepsané podmínky pro komunikaci HEAppE s lokálním simulovaným výpočetním clusterem.

Cílem tohoto rozšíření je přinést uživatelům možnost vyzkoušet si HEAppE Middleware na svém počítači bez nutnosti napojení na fyzický superpočítač. Proto je i komunikace s lokálním simulovaným výpočetním clusterem probíhá stejně, jako je tomu u skutečných výpočetních clusterů. Využívá se stejných protokolů, metod šifrování nebo postupů při autentizaci. To vše proto aby byl následný přechod z lokálně simulovaného prostředí na skutečné superpočítače co nejsnazší.

Dále bylo navrženo a implementováno rozšíření o funkcionalitu pro vytváření uživatelsky definovaných šablon. Prerekvizitou je existence generické šablony, která slouží především k testovacím účelům. Z této generické šablony se pak vychází při definování uživatelské šablony, která je následně uložena persistentně do systému HEAppE. A koncový uživatel ji může libovolně využívat.

Všechny nově přidané funkcionality byly důkladně otestovány, současný stav projektu s těmito změnami umožňuje publikaci do nové verze HEAppE Middleware. Současně byl vytvořen testovací scénář pro automatizované testování, který je možné spouštět při každé publikaci změny do repozitáře projektu.

Věřím, že díky výše popsaným funkcionalitám bude dále systém HEAppE Middleware rozšiřován do superpočítačových center, především díky rozšíření o možnost lokálního spouštění úloh. Uživatelé si budou moct HEAppE před nasazením v daném superpočítačovém centru vyzkoušet na svém počítači bez nutnosti přístupu na skutečné clustery. Jedním z účelů přidání této funkcionality je podpora popularizace HEAppE Middleware.
\endinput